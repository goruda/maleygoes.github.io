\documentclass[10pt]{article}
\usepackage{amsmath}
\addtolength{\jot}{1.5em} %spacing out gathered equations
%\usepackage{geometry}
\usepackage{amssymb}
\usepackage[default]{lato}
\usepackage[utf8]{inputenc}
\usepackage[english]{babel}
\usepackage{amsthm}
\usepackage{tikz}
\usepackage{url}
\usepackage{graphicx}
\usepackage{fancyhdr}
%\pagestyle{fancy}
\usepackage{tcolorbox}
\usepackage{hyperref}

\newcommand*\circled[1]{\tikz[baseline=(char.base)]{
		\node[shape=circle,color=red,draw,inner sep=2pt] (char) {#1};}}

\begin{document}
	\thispagestyle{empty}
\section{Worked Example}
\subsection{From Section 3.2}

\textbf{Exercise}: Consider the one-parameter family of functions given by $q(x) = \dfrac{4e^{-x}}{x-2c}$ where $c>0$.\\

\begin{enumerate}
	\item $q(x)$ has a vertical asymptote at $x=2c-1$ True or false?
		\subitem (ANS) False.  $ q(x) $ would have an asymptote at $x=2c$.
		
	\item Evaluating limits:  $\displaystyle\lim_{x\to \infty} q(x)$.  
		\subitem (ANS) We see the numerator is a decaying exponential, $4e^{-x}$ and the denominator is an increasing linear function $ x-2c $.  Exponentials always dominate linear functions in the long run.  As x gets huge, the numerator will be rushing to zero and the denominator will be increasing.  Therefore this $L = 0$.
	
	\item Evaluate $\displaystyle\lim_{x\to -\infty} q(x)$. 
		\subitem(ANS) Here, the reverse of what we said above is true.  As $x\to -\infty$ we see the top is getting to huge positive numbers and dominating the linear function in the bottom.  But the value of the denominator is negative, so this will drive the whole expression to negative infinity.
		
	\item Find $q'(x)$.  
		\subitem (ANS)Here we have a quotient so we can apply the quotient rule.  Don't forget the chain rule on the $4e^{-x}$.
		$$\begin{gathered}
			q = \frac{4e^{-x}}{x-2c}\\
			q' = \frac{[4e^{-x}(-1)(x-2c)] - [4e^{-x}(1)]}{(x-2c)^2}\\
			q ' = \frac{-4e^{-x}(x-2c+1)}{(x-2c)^2}
		\end{gathered}$$
	
	Notice I factored a $-4e^{-x}$ which changed the sign of the term on the right side of the numerator.  \textbf{On an exam, you could leave it like line \#2}, without the last simplification, and I would accept it as is.
	
	
	\item The function $ q(x) $ has exactly one critical point.  
		\subitem (ANS) Well, the only place we can get $ q'(x)=0$ OR undefined would be what makes the top = 0, or what makes the bottom =0.  Since we already said $c>0$ we can throw that out.\\
		
		So we can just find what makes the top = 0.  Solving $$-4e^{-x}(x-2c+1)=0$$ gets us to $x = 2c - 1$.  (Divide both sides by $-4e^{-x}$ and just solve the simpler equation for x).
		
		
	\item Is that CP a max or min or neither?
		\subitem I don't want to take the second derivative of this, do you?  So I'd probably stick with just testing the sign of q' before and after $x=2c-1$.  Students get confused here because the values you choose are ``up to you" which we are maybe not used to.  You could choose $x=2c-2$ and $x=2c-0.5$ because these are just before and after $x=2c-1$.  (Why can't we use $ x=2c $? It is a vertical asymptote so the function breaks here).\\
		
		Another thing to think about: if this works for all $c>0$, then shouldn't it work just the same for $c=1$ as it does for all other c-values?  Yes.\\
		
		So let $ c=1 $ and see what happens at this critical point.  $x=2(1)-2=0$ and $x=2(1)-0.5=1.5$\\
		
		You can just plug these into your calculator and see that it does in fact go from $q'>0 \to q'<0$ which makes it a \textbf{local max}.\\
		
		\textbf{Desmos Visualization}: \url{https://www.desmos.com/calculator/w555jrgsdu} For a visualization of this.  Notice how you can play with C to see how things shift but stay the same.
	
\end{enumerate}
\vspace{2em}





\end{document}